\documentclass{article}
\usepackage[utf8]{inputenc}
\usepackage[french]{babel}
\usepackage[T1]{fontenc}
\PassOptionsToPackage{hyphens}{url}\usepackage{hyperref}

\title{\textbf{\huge{Projet Élections}}}
\author{NG Thibault \& DU Camille}

\begin{document}

    \maketitle

    \newpage
    \tableofcontents

    \newpage
    \section{Importation des données}



    \newpage
    \section{Interface graphique}

    \subsection{Carte}

    Parmis les données auxquelles nous avions accès, nous avions uniquement l'état ainsi que le numéro du district.

    Les données que l'on a trouvé en termes géographique ne contenaient que les comtés.

    Il a donc fallut trouver des données reliant comtés et numéro de district

    En combinant les deux et en fusionnant les comtés en un plus gros district nous avons une carte des districts.




    \section{Sources}\label{sec:sources}



    \nocite{*}
    \bibliographystyle{unsrt}
    \bibliography{sources}

\end{document}
